% Gemini theme
% https://github.com/anishathalye/gemini
%
% We try to keep this Overleaf template in sync with the canonical source on
% GitHub, but it's recommended that you obtain the template directly from
% GitHub to ensure that you are using the latest version.

\documentclass[final]{beamer}

% ====================
% Packages
% ====================

\usepackage[T1]{fontenc}
\usepackage{lmodern}
\usepackage{beamerposter}
\geometry{papersize={841mm,1188mm}}
\usetheme{gemini}
\usecolortheme{gemini}
\usepackage{graphicx}
\usepackage{booktabs}
\usepackage{tikz}
\usepackage{pgfplots}
\usepackage{array}
\pgfplotsset{compat=1.14}
\renewcommand{\d}{\textrm{d}}
% ====================
% Lengths
% ====================

% If you have N columns, choose \sepwidth and \colwidth such that
% (N+1)*\sepwidth + N*\colwidth = \paperwidth
\newlength{\sepwidth}
\newlength{\colwidth}
\setlength{\sepwidth}{0.025\paperwidth}
\setlength{\colwidth}{0.3\paperwidth}

\newcommand{\separatorcolumn}{\begin{column}{\sepwidth}\end{column}}

% ====================
% Title
% ====================

\title{Ab Initio Study into the Reorientation Rates of NnVH Defects}
\title{Ab Initio Study of Hydrogen Quantum Tunnelling in NnVH Defects}


\author{Samuel J. Frost}

\institute[shortinst]{University of Warwick}

% ====================
% Footer (optional)
% ====================

\footercontent{
  \href{https://www.samuelfrost.co.uk}{https://www.samuelfrost.co.uk} \hfill
  \href{mailto:alyssa.p.hacker@example.com}{Samuel.Frost@Warwick.ac.uk}}
% (can be left out to remove footer)

% ====================
% Logo (optional)
% ====================

% use this to include logos on the left and/or right side of the header:
% \logoright{\includegraphics[height=7cm]{logo1.pdf}}
% \logoleft{\includegraphics[height=7cm]{logo2.pdf}}

% ====================
% Body
% ====================

\begin{document}

\begin{frame}[t]
\begin{columns}[t]
\separatorcolumn

\begin{column}{\colwidth}
    
  \begin{block}{NnVH Defects in Diamond}
    \begin{itemize}
        \item NnVH defects contain a central vacancy, surrounded by $n$ substitutional nitrogens, where $n \in [0, 3]$, as well as a hydrogen sitting in the vacancy.
        \item The original proposed structure of these defects were the hydrogen being bonded to one of the unsaturated carbons surrounding the vacancy, giving rise to a low order point group. (NOT REALLY)
        \item EPR experiments observe a higher order point group, leading to the suspicion that the hydrogen was bonded to one of the nitrogens, however ab initio studies show that this is too energetically unfavourable.
        \item This lead to the theory that the hydrogen is rapidly reorientating \emph{between} carbon atoms at a rate fast enough that it shows a higher order averaged symmetry when observed under EPR.
        \item Classical estimates for the reorientation give incredibly low rates for reorientation, with hydrogen staying at one carbon site for over 300s, much too slow to show averaged symmetry under EPR, which measures at rates in the GHz range. This is shown below in equation .
        \item This gave rise to the now widely accepted theory that the hydrogen is in fact \emph{quantum tunnelling} between equivalent carbon sites.
    \end{itemize}

    
    \begin{figure}
      \centering
        \includegraphics[width=\textwidth]{n2vh_coloured.png}
      \caption{N$_2$VH in its averaged C$_{2v}$ symmetry position, switching rapidly between the two central carbons. Hydrogen is shown in white, nitrogen in blue, and carbon in grey, most of the surrounding structure has been omitted for clarity. Generated using OVITO.}
    \end{figure}



  \end{block}

  \begin{block}{Attempt Frequency from Phonon Modes}
    \begin{itemize}
        \item In order to calculator the tunnelling rate, the attempt frequency over the barrier must be known.
        \item This can be approximated as the vibrational frequency of hydrogen in the \emph{direction} of the reaction coordinate, $\nu$.
        \item \textbf{Finite displacement phonon calculations} were performed to calculate the phonon modes pertaining to each atom. A $3N\times3N$ \emph{dynamical matrix} is retrieved, which contains the force acted on each atom due to the displacement of another when moved a finite displacement from its minimum energy point.
        \item This can be represented mathematically by equation , where $E$ is energy, $r_n$ is the displacement(?) of an atom and $m_n$ is its corresponding mass, where $i$ and $j$ occur for every possible pair of atoms.
    \end{itemize}
    \begin{align*}
        \frac{{\d}^2E}{{\d}r_i\,{\d}r_j}\frac{1}{\sqrt{m_i\,m_j}}
    \end{align*}
    \begin{itemize}
        \item From here we can diagonalise the matrix to retrieve its eigenvalues and eigenvectors, which detail the strength and direction of each phonon mode.
        \item The dot of product of these modes, with the normalised direction of the reaction coordinate will give the tunnelling frequency, $\nu$.
        \item Due to the computational complexity of this take, only the calculation for the N$_2$VH defect was performed, however the vibrational frequency along the path was found to be $\nu = 1363.5$\,cm$^{-1}$.
        \item This is remarkably close to the one phonon mode of the diamond lattice, $1332$\,cm$^{-1}$, a value which has previously been used as an estimate for the attempt frequency, giving a good justification for its further use in this work. Using either value gives no meaningful difference to the results for N$_2$VH.
    \end{itemize}

  \end{block}

  \begin{alertblock}{Computational Details}
    \begin{itemize}
        \item All calculations were performed using CASTEP 23.1, using a periodic 64 atom simulation cell.
        \item The meta-GGA RSCAN functional was used for all calculations unless stated otherwise. 
        \item A plane wave cut-off of $1\,000$ eV and a $4\times 4\times 4$ monkhorst-pack k-point grid was used, with a force tolerance of 0.02 eV / {\AA} for geometrical minimisation.
        \item A lattice constant of 3.5537 was found during convergence testing, close to the experimental value of 3.567.
    \end{itemize}

  \end{alertblock}

\end{column}

\separatorcolumn

\begin{column}{\colwidth}

  \begin{block}{Probability of Tunnelling}
    % To study the energy barrier that is required to be overcame in order for the hydrogen to tunnel, nudged elastic band (NEB) calculations were performed. This takes the form of multiple \emph{images} between an initial and final state connected by \emph{springs}, such that an equal force is applied between each image. This ensures that during the minimisation procedure each image does not simply minimise to the lowest possible state. This finds the \emph{minimum energy path}, between the two carbons which the hydrogen is tunnelling between, giving a reaction coordinate for the quantum tunnelling. Due to the computational intensity of these calculations, only a small number of points along the reaction coordinate can be sampled, in order to fill in the gaps, Gaussian Process Regression (GPR) can be employed to interpolate the data.  

    \begin{itemize}
      \item \textbf{Nudged Elastic Band (NEB)} calculations were performed to find the energy along the reaction coordinate of the tunnelling hydrogen for each NnVH system.
      \item \textbf{Gaussian Process Regression} was employed to interpolate between the discrete points along the reaction coordinate such that continuous integration could be performed later.
      \item \textbf{The WKB approximation} was then used to calculate the instantaneous probability of tunnelling.  
    \end{itemize}
    \begin{align*}        
        P &= \exp \left(\frac{-4\pi}{h}\int_a^b \sqrt{2m\,(V-E)}  \right)
    \end{align*}
    In this equation, the integral is performed between the \emph{turning points} of the energy barrier, $a$ and $b$, where the potential and energy of the system are equal, such that $V = E$. 
    It is not \emph{only} the hydrogen that moves when quantum tunnelling is taking place, the atoms neighbouring the vacancy also account for a small amount of movement (around 10\% of the total displacement).
    Therefore, when calculating the mass, we can consider their mass in proportion to their contribution to the reaction coordinate. 
    \begin{align*}
        \frac{1}{m^*} = \sum_i \frac{1}{m_i} \frac{\Delta r_i}{q} 
    \end{align*}
    Where $m^*$ is the effective mass, $\Delta r_i$ is the total displacement of atom $i$, and $q$ is the total displacement of the reaction coordinate, such that $\sum_i \frac{\Delta r_i}{q} = 1$

    Really, we need to go ahead and calc dx/dq, and do $1/μ = Σ(1/mi) * (∂xi/∂q)^2$
    We just decided that the number for distance that ASE gives us is too strange to reliably use since we can't reproduce it 
    Actually I think our errors might be from pbc -> use find$\_$mic maybe? copy from source
    TODO: USE FIND-MIC TO GET REPRODUCABLE PATH VALUE
          USE THAT VALUE TO THEN GET THE REDUCED MASS OF TUNNELLING BY FINDING DX/DQ
          LOOK INTO FIXING THE TEMPERATURE RESULTS, BUT 0K MAY JUST HAVE TO BE FINE
    
  \end{block}

  \begin{block}{Energy Levels}

    Although anharmonic effects are most likely present, the hydrogen atom can be approximated as a quantum harmonic oscillator (QHO), this means that the ground state energy (\emph{i.e.} the energy at 0K) would be 
    $$E = \frac{1}{2}\,h\nu$$
    We have multiple phonon modes, with different strengths pointing in the direction of the hydrogen atom. Taking the sum of all these modes can give us an estimate for the vibrational frequency in that direction. 
    $$ \sum_i \frac{1}{2}\,h\nu_i + h\nu_i \frac{1}{\exp^{h_i\nu/kBT} - 1} $$

  \end{block}

  \begin{block}{}


\begin{figure}[htbp]
    \centering
    % Set the width of the figure

    \resizebox{\textwidth}{!}{
    % This file was created with tikzplotlib v0.10.1.
\begin{tikzpicture}[font=\fontsize{6}{7}\selectfont]
% This file was created with tikzplotlib v0.10.1.
\definecolor{darkslategray38}{RGB}{38,38,38}
\definecolor{lavender234234242}{RGB}{234,234,242}
\definecolor{mediumseagreen85168104}{RGB}{85,168,104}
\definecolor{steelblue76114176}{RGB}{76,114,176}

\begin{axis}[
axis background/.style={fill=lavender234234242},
axis line style={white},
legend cell align={left},
legend style={
  fill opacity=0.8,
  draw opacity=1,
  text opacity=1,
  at={(0.5,0.09)},
  anchor=south,
  draw=none,
  fill=lavender234234242
},
tick align=outside,
tick pos=left,
x grid style={white},
xlabel=\textcolor{darkslategray38}{\fontsize{8}{9}\selectfont Reaction Coordinate / {\AA} },
xmajorgrids,
xmin=-0.0540455025233226, xmax=1.13495555298978,
xtick style={color=darkslategray38},
y grid style={white},
ylabel=\textcolor{darkslategray38}{\fontsize{8}{9}\selectfont Barrier Height / eV},
ymajorgrids,
ymin=-0.0218074212401964, ymax=0.430657530668547,
ytick style={color=darkslategray38}
]
\addplot [draw=steelblue76114176, fill=steelblue76114176, mark=*, only marks]
table{%
x  y
0 0
0.145478959605773 0.0421025100004044
0.932704224327079 0.044033109999873
0.331610559783429 0.232790449999811
0.746958798653657 0.236188139999285
0.538821061179613 0.409716329999355
1.08091005046645 -1.07000050775241e-06
};
\addlegendentry{Observations}
\path [fill=mediumseagreen85168104, fill opacity=0.5, line width=0.12pt]
(axis cs:0,1.96005085623076e-05)
--(axis cs:0,-1.9599488979334e-05)
--(axis cs:0.00434100421874077,-0.000446509521810643)
--(axis cs:0.00868200843748155,-0.000785915387606347)
--(axis cs:0.0130230126562223,-0.00101844037290698)
--(axis cs:0.0173640168749631,-0.00114389750873404)
--(axis cs:0.0217050210937039,-0.00116220591547595)
--(axis cs:0.0260460253124446,-0.00107335041705555)
--(axis cs:0.0303870295311854,-0.000877374090487393)
--(axis cs:0.0347280337499262,-0.000574374902621239)
--(axis cs:0.039069037968667,-0.000164503237289819)
--(axis cs:0.0434100421874077,0.000352040318129305)
--(axis cs:0.0477510464061485,0.000975007090753515)
--(axis cs:0.0520920506248893,0.00170410239221385)
--(axis cs:0.0564330548436301,0.00253898755221495)
--(axis cs:0.0607740590623708,0.00347928191416951)
--(axis cs:0.0651150632811116,0.00452456478663674)
--(axis cs:0.0694560674998524,0.00567437734194427)
--(axis cs:0.0737970717185931,0.00692822445256135)
--(axis cs:0.0781380759373339,0.0082855764553976)
--(axis cs:0.0824790801560747,0.00974587083408931)
--(axis cs:0.0868200843748155,0.0113085138092815)
--(axis cs:0.0911610885935562,0.0129728818270778)
--(axis cs:0.095502092812297,0.0147383229355073)
--(axis cs:0.0998430970310378,0.0166041580385663)
--(axis cs:0.104184101249779,0.0185696820161187)
--(axis cs:0.108525105468519,0.0206341646950407)
--(axis cs:0.11286610968726,0.0227968516499548)
--(axis cs:0.117207113906001,0.0250569647946369)
--(axis cs:0.121548118124742,0.0274137026785673)
--(axis cs:0.125889122343482,0.0298662402620682)
--(axis cs:0.130230126562223,0.0324137274406657)
--(axis cs:0.134571130780964,0.0350552832821397)
--(axis cs:0.138912134999705,0.0377899669199299)
--(axis cs:0.143253139218446,0.0406164104541902)
--(axis cs:0.147594143437186,0.0431788839579744)
--(axis cs:0.151935147655927,0.0454669882080134)
--(axis cs:0.156276151874668,0.0478591618617882)
--(axis cs:0.160617156093409,0.0503584499496212)
--(axis cs:0.164958160312149,0.0529672361835934)
--(axis cs:0.16929916453089,0.0556876387236719)
--(axis cs:0.173640168749631,0.0585215333653126)
--(axis cs:0.177981172968372,0.061470552803083)
--(axis cs:0.182322177187112,0.0645360821225574)
--(axis cs:0.186663181405853,0.0677192536990135)
--(axis cs:0.191004185624594,0.0710209422705625)
--(axis cs:0.195345189843335,0.0744417604301025)
--(axis cs:0.199686194062076,0.077982054633349)
--(axis cs:0.204027198280816,0.081641901770147)
--(axis cs:0.208368202499557,0.0854211063263424)
--(axis cs:0.212709206718298,0.0893191981543445)
--(axis cs:0.217050210937039,0.0933354308652096)
--(axis cs:0.221391215155779,0.0974687808519841)
--(axis cs:0.22573221937452,0.101717946951117)
--(axis cs:0.230073223593261,0.106081350747001)
--(axis cs:0.234414227812002,0.110557137522385)
--(axis cs:0.238755232030743,0.11514317785559)
--(axis cs:0.243096236249483,0.119837069863636)
--(axis cs:0.247437240468224,0.124636142088548)
--(axis cs:0.251778244686965,0.129537457022196)
--(axis cs:0.256119248905706,0.134537815263355)
--(axis cs:0.260460253124446,0.139633760298756)
--(axis cs:0.264801257343187,0.144821583898125)
--(axis cs:0.269142261561928,0.150097332111341)
--(axis cs:0.273483265780669,0.155456811854102)
--(axis cs:0.27782426999941,0.16089559806623)
--(axis cs:0.28216527421815,0.166409041424873)
--(axis cs:0.286506278436891,0.171992276591942)
--(axis cs:0.290847282655632,0.177640230971523)
--(axis cs:0.295188286874373,0.18334763394712)
--(axis cs:0.299529291093113,0.189109026557125)
--(axis cs:0.303870295311854,0.194918771541777)
--(axis cs:0.308211299530595,0.200771063629753)
--(axis cs:0.312552303749336,0.206659939736555)
--(axis cs:0.316893307968076,0.212579288035158)
--(axis cs:0.321234312186817,0.218522851471875)
--(axis cs:0.325575316405558,0.224484196034405)
--(axis cs:0.329916320624299,0.230456013525651)
--(axis cs:0.33425732484304,0.236058915250833)
--(axis cs:0.33859832906178,0.241421244187183)
--(axis cs:0.342939333280521,0.246778862242976)
--(axis cs:0.347280337499262,0.252128141060039)
--(axis cs:0.351621341718003,0.25746501295747)
--(axis cs:0.355962345936743,0.262785284350401)
--(axis cs:0.360303350155484,0.268084667572434)
--(axis cs:0.364644354374225,0.273358790343849)
--(axis cs:0.368985358592966,0.27860320135042)
--(axis cs:0.373326362811707,0.283813375029352)
--(axis cs:0.377667367030447,0.288984716323074)
--(axis cs:0.382008371249188,0.294112565635384)
--(axis cs:0.386349375467929,0.299192204072676)
--(axis cs:0.39069037968667,0.304218859000841)
--(axis cs:0.39503138390541,0.309187709927352)
--(axis cs:0.399372388124151,0.314093894708821)
--(axis cs:0.403713392342892,0.318932516078733)
--(axis cs:0.408054396561633,0.323698648487131)
--(axis cs:0.412395400780374,0.328387345241884)
--(axis cs:0.416736404999114,0.332993645939154)
--(axis cs:0.421077409217855,0.33751258416916)
--(axis cs:0.425418413436596,0.341939195482084)
--(axis cs:0.429759417655337,0.346268525597608)
--(axis cs:0.434100421874077,0.350495638840248)
--(axis cs:0.438441426092818,0.354615626781701)
--(axis cs:0.442782430311559,0.358623617070122)
--(axis cs:0.4471234345303,0.362514782425426)
--(axis cs:0.45146443874904,0.366284349778585)
--(axis cs:0.455805442967781,0.369927609532103)
--(axis cs:0.460146447186522,0.373439924917846)
--(axis cs:0.464487451405263,0.376816741427869)
--(axis cs:0.468828455624004,0.380053596292895)
--(axis cs:0.473169459842744,0.38314612798249)
--(axis cs:0.477510464061485,0.38609008570054)
--(axis cs:0.481851468280226,0.388881338848561)
--(axis cs:0.486192472498967,0.391515886428965)
--(axis cs:0.490533476717707,0.393989866359318)
--(axis cs:0.494874480936448,0.396299564667056)
--(axis cs:0.499215485155189,0.398441424531484)
--(axis cs:0.50355648937393,0.400412055134292)
--(axis cs:0.507897493592671,0.402208240267356)
--(axis cs:0.512238497811411,0.403826946615952)
--(axis cs:0.516579502030152,0.405265331550027)
--(axis cs:0.520920506248893,0.40652074998571)
--(axis cs:0.525261510467634,0.407590758825736)
--(axis cs:0.529602514686374,0.408473111974758)
--(axis cs:0.533943518905115,0.409165689361792)
--(axis cs:0.538284523123856,0.409662929254693)
--(axis cs:0.542625527342597,0.409440087162348)
--(axis cs:0.546966531561337,0.408945866716042)
--(axis cs:0.551307535780078,0.408260768899654)
--(axis cs:0.555648539998819,0.407386769682174)
--(axis cs:0.55998954421756,0.406325902243668)
--(axis cs:0.564330548436301,0.405080393242434)
--(axis cs:0.568671552655041,0.403652674089885)
--(axis cs:0.573012556873782,0.402045377973639)
--(axis cs:0.577353561092523,0.400261333620702)
--(axis cs:0.581694565311264,0.398303557912093)
--(axis cs:0.586035569530004,0.396175247926362)
--(axis cs:0.590376573748745,0.393879772619167)
--(axis cs:0.594717577967486,0.391420664233704)
--(axis cs:0.599058582186227,0.388801609498371)
--(axis cs:0.603399586404967,0.386026440652073)
--(axis cs:0.607740590623708,0.383099126331134)
--(axis cs:0.612081594842449,0.380023762348834)
--(axis cs:0.61642259906119,0.376804562396335)
--(axis cs:0.620763603279931,0.373445848693252)
--(axis cs:0.625104607498671,0.369952042615097)
--(axis cs:0.629445611717412,0.366327655324227)
--(axis cs:0.633786615936153,0.362577278430294)
--(axis cs:0.638127620154894,0.358705574705654)
--(axis cs:0.642468624373635,0.354717268880342)
--(axis cs:0.646809628592375,0.35061713854055)
--(axis cs:0.651150632811116,0.346410005153829)
--(axis cs:0.655491637029857,0.342100725243231)
--(axis cs:0.659832641248598,0.337694181731771)
--(axis cs:0.664173645467338,0.333195275477558)
--(axis cs:0.668514649686079,0.328608917018994)
--(axis cs:0.67285565390482,0.323940018548239)
--(axis cs:0.677196658123561,0.319193486129952)
--(axis cs:0.681537662342302,0.314374212181233)
--(axis cs:0.685878666561042,0.309487068227205)
--(axis cs:0.690219670779783,0.304536897945327)
--(axis cs:0.694560674998524,0.299528510509791)
--(axis cs:0.698901679217265,0.294466674245513)
--(axis cs:0.703242683436005,0.289356110598347)
--(axis cs:0.707583687654746,0.284201488424562)
--(axis cs:0.711924691873487,0.27900741859503)
--(axis cs:0.716265696092228,0.273778448895996)
--(axis cs:0.720606700310968,0.268519059174364)
--(axis cs:0.724947704529709,0.263233656585399)
--(axis cs:0.72928870874845,0.257926570513716)
--(axis cs:0.733629712967191,0.252602045613957)
--(axis cs:0.737970717185932,0.247264225417188)
--(axis cs:0.742311721404672,0.241917062644108)
--(axis cs:0.746652725623413,0.236558098560646)
--(axis cs:0.750993729842154,0.230635290267408)
--(axis cs:0.755334734060895,0.224667669821254)
--(axis cs:0.759675738279635,0.218711801656941)
--(axis cs:0.764016742498376,0.21277432264597)
--(axis cs:0.768357746717117,0.206861612973364)
--(axis cs:0.772698750935858,0.200979906398513)
--(axis cs:0.777039755154598,0.19513529470141)
--(axis cs:0.781380759373339,0.189333719983894)
--(axis cs:0.78572176359208,0.183580964629206)
--(axis cs:0.790062767810821,0.177882640849771)
--(axis cs:0.794403772029562,0.172244180359554)
--(axis cs:0.798744776248302,0.166670824364984)
--(axis cs:0.803085780467043,0.161167613963186)
--(axis cs:0.807426784685784,0.155739380998197)
--(axis cs:0.811767788904525,0.150390739409836)
--(axis cs:0.816108793123265,0.145126077102089)
--(axis cs:0.820449797342006,0.139949548353503)
--(axis cs:0.824790801560747,0.134865066789003)
--(axis cs:0.829131805779488,0.129876298930119)
--(axis cs:0.833472809998228,0.124986658338709)
--(axis cs:0.837813814216969,0.120199300367103)
--(axis cs:0.84215481843571,0.115517117526177)
--(axis cs:0.846495822654451,0.110942735480427)
--(axis cs:0.850836826873192,0.106478509678008)
--(axis cs:0.855177831091932,0.102126522621313)
--(axis cs:0.859518835310673,0.0978885817821217)
--(axis cs:0.863859839529414,0.0937662181636244)
--(axis cs:0.868200843748155,0.0897606855094136)
--(axis cs:0.872541847966895,0.0858729601581092)
--(axis cs:0.876882852185636,0.0821037415398394)
--(axis cs:0.881223856404377,0.0784534533091876)
--(axis cs:0.885564860623118,0.0749222451063229)
--(axis cs:0.889905864841859,0.0715099949356462)
--(axis cs:0.894246869060599,0.0682163121464442)
--(axis cs:0.89858787327934,0.0650405409932096)
--(axis cs:0.902928877498081,0.0619817647387821)
--(axis cs:0.907269881716822,0.0590388102290128)
--(axis cs:0.911610885935562,0.0562102527725016)
--(axis cs:0.915951890154303,0.0534944208465179)
--(axis cs:0.920292894373044,0.0508893988552931)
--(axis cs:0.924633898591785,0.0483930186223193)
--(axis cs:0.928974902810526,0.0460027439138837)
--(axis cs:0.933315907029266,0.043608453387361)
--(axis cs:0.937656911248007,0.0406749363327815)
--(axis cs:0.941997915466748,0.0378303776123113)
--(axis cs:0.946338919685489,0.0350785140698837)
--(axis cs:0.950679923904229,0.0324204141449811)
--(axis cs:0.95502092812297,0.0298570469709652)
--(axis cs:0.959361932341711,0.0273893439877725)
--(axis cs:0.963702936560452,0.0250182087080268)
--(axis cs:0.968043940779193,0.0227445190680872)
--(axis cs:0.972384944997933,0.0205691279201557)
--(axis cs:0.976725949216674,0.0184928628425533)
--(axis cs:0.981066953435415,0.0165165256038616)
--(axis cs:0.985407957654156,0.0146408913989941)
--(axis cs:0.989748961872896,0.0128667079074392)
--(axis cs:0.994089966091637,0.0111946941998168)
--(axis cs:0.998430970310378,0.00962553950868976)
--(axis cs:1.00277197452912,0.00815990187633561)
--(axis cs:1.00711297874786,0.00679840668990422)
--(axis cs:1.0114539829666,0.00554164511423822)
--(axis cs:1.01579498718534,0.00439017243211518)
--(axis cs:1.02013599140408,0.00334450630186549)
--(axis cs:1.02447699562282,0.00240512494226004)
--(axis cs:1.02881799984156,0.00157246525460126)
--(axis cs:1.0331590040603,0.000846920892007387)
--(axis cs:1.03750000827904,0.000228840285128085)
--(axis cs:1.04184101249779,-0.000281475366862715)
--(axis cs:1.04618201671653,-0.000683774133736777)
--(axis cs:1.05052302093527,-0.000977855418529266)
--(axis cs:1.05486402515401,-0.00116357205972056)
--(axis cs:1.05920502937275,-0.00124083251707174)
--(axis cs:1.06354603359149,-0.00120960330057425)
--(axis cs:1.06788703781023,-0.00106991225560307)
--(axis cs:1.07222804202897,-0.000821855811811426)
--(axis cs:1.07656904624771,-0.000465638238642895)
--(axis cs:1.08091005046645,-2.06695279067983e-05)
--(axis cs:1.08091005046645,1.85304709948666e-05)
--(axis cs:1.08091005046645,1.85304709948666e-05)
--(axis cs:1.07656904624771,0.00117074655206963)
--(axis cs:1.07222804202897,0.00230287555805602)
--(axis cs:1.06788703781023,0.00339946136755978)
--(axis cs:1.06354603359149,0.00446432042962338)
--(axis cs:1.05920502937275,0.00550143817100123)
--(axis cs:1.05486402515401,0.00651492369655165)
--(axis cs:1.05052302093527,0.00750899616676262)
--(axis cs:1.04618201671653,0.00848797481500264)
--(axis cs:1.04184101249779,0.00945626950341403)
--(axis cs:1.03750000827904,0.0104183712118359)
--(axis cs:1.0331590040603,0.0113788423087737)
--(axis cs:1.02881799984156,0.0123423065681392)
--(axis cs:1.02447699562282,0.0133134389312613)
--(axis cs:1.02013599140408,0.0142969550273224)
--(axis cs:1.01579498718534,0.0152976004710281)
--(axis cs:1.0114539829666,0.0163201399598967)
--(axis cs:1.00711297874786,0.0173693461949463)
--(axis cs:1.00277197452912,0.0184499886502624)
--(axis cs:0.998430970310378,0.0195668222178396)
--(axis cs:0.994089966091637,0.020724575754974)
--(axis cs:0.989748961872896,0.0219279405624412)
--(axis cs:0.985407957654156,0.0231815588223849)
--(axis cs:0.981066953435415,0.0244900120261859)
--(axis cs:0.976725949216674,0.0258578094235553)
--(axis cs:0.972384944997933,0.0272893765271196)
--(axis cs:0.968043940779193,0.0287890437107531)
--(axis cs:0.963702936560452,0.0303610349507694)
--(axis cs:0.959361932341711,0.0320094567838328)
--(axis cs:0.95502092812297,0.0337382876238519)
--(axis cs:0.950679923904229,0.0355513677982731)
--(axis cs:0.946338919685489,0.0374523915060811)
--(axis cs:0.941997915466748,0.0394449062792409)
--(axis cs:0.937656911248007,0.0415323644050597)
--(axis cs:0.933315907029266,0.043720681974771)
--(axis cs:0.928974902810526,0.046640008614635)
--(axis cs:0.924633898591785,0.0497568742065728)
--(axis cs:0.920292894373044,0.0529626643590504)
--(axis cs:0.915951890154303,0.0562561069271828)
--(axis cs:0.911610885935562,0.0596360460133541)
--(axis cs:0.907269881716822,0.0631013178434223)
--(axis cs:0.902928877498081,0.0666507339383412)
--(axis cs:0.89858787327934,0.0702830759044982)
--(axis cs:0.894246869060599,0.0739970925500757)
--(axis cs:0.889905864841859,0.0777914975788827)
--(axis cs:0.885564860623118,0.0816649674075173)
--(axis cs:0.881223856404377,0.0856161389635489)
--(axis cs:0.876882852185636,0.0896436074169243)
--(axis cs:0.872541847966895,0.0937459238304982)
--(axis cs:0.868200843748155,0.0979215927292782)
--(axis cs:0.863859839529414,0.102169069593933)
--(axis cs:0.859518835310673,0.106486758287968)
--(axis cs:0.855177831091932,0.110873008429352)
--(axis cs:0.850836826873192,0.115326112719163)
--(axis cs:0.846495822654451,0.119844304240252)
--(axis cs:0.84215481843571,0.124425753740156)
--(axis cs:0.837813814216969,0.129068566912622)
--(axis cs:0.833472809998228,0.133770781692944)
--(axis cs:0.829131805779488,0.13853036558267)
--(axis cs:0.824790801560747,0.143345213019271)
--(axis cs:0.820449797342006,0.148213142807294)
--(axis cs:0.816108793123265,0.153131895627027)
--(axis cs:0.811767788904525,0.158099131637573)
--(axis cs:0.807426784685784,0.163112428191002)
--(axis cs:0.803085780467043,0.168169277674724)
--(axis cs:0.798744776248302,0.173267085499467)
--(axis cs:0.794403772029562,0.178403168250896)
--(axis cs:0.790062767810821,0.183574752024294)
--(axis cs:0.78572176359208,0.188778970964298)
--(axis cs:0.781380759373339,0.194012866037731)
--(axis cs:0.777039755154598,0.199273384081821)
--(axis cs:0.772698750935858,0.204557377206336)
--(axis cs:0.768357746717117,0.20986160273172)
--(axis cs:0.764016742498376,0.215182724185818)
--(axis cs:0.759675738279635,0.220517315273668)
--(axis cs:0.755334734060895,0.22586187660648)
--(axis cs:0.750993729842154,0.231212958840487)
--(axis cs:0.746652725623413,0.236616847006479)
--(axis cs:0.742311721404672,0.242582114573904)
--(axis cs:0.737970717185932,0.248546095406482)
--(axis cs:0.733629712967191,0.254495557921137)
--(axis cs:0.72928870874845,0.260423547891285)
--(axis cs:0.724947704529709,0.266323183764512)
--(axis cs:0.720606700310968,0.27218758335638)
--(axis cs:0.716265696092228,0.278009864224953)
--(axis cs:0.711924691873487,0.283783153834879)
--(axis cs:0.707583687654746,0.289500601923617)
--(axis cs:0.703242683436005,0.295155393481672)
--(axis cs:0.698901679217265,0.300740761882159)
--(axis cs:0.694560674998524,0.306250001981102)
--(axis cs:0.690219670779783,0.311676483099043)
--(axis cs:0.685878666561042,0.317013661827752)
--(axis cs:0.681537662342302,0.322255094618716)
--(axis cs:0.677196658123561,0.327394450117113)
--(axis cs:0.67285565390482,0.332425521208102)
--(axis cs:0.668514649686079,0.337342236744789)
--(axis cs:0.664173645467338,0.342138672928781)
--(axis cs:0.659832641248598,0.346809064315834)
--(axis cs:0.655491637029857,0.351347814420403)
--(axis cs:0.651150632811116,0.355749505894426)
--(axis cs:0.646809628592375,0.360008910256917)
--(axis cs:0.642468624373635,0.364120997152419)
--(axis cs:0.638127620154894,0.368080943117921)
--(axis cs:0.633786615936153,0.37188413983927)
--(axis cs:0.629445611717412,0.37552620187976)
--(axis cs:0.625104607498671,0.37900297386505)
--(axis cs:0.620763603279931,0.38231053711043)
--(axis cs:0.61642259906119,0.385445215677989)
--(axis cs:0.612081594842449,0.388403581852985)
--(axis cs:0.607740590623708,0.391182461030573)
--(axis cs:0.603399586404967,0.393778936005863)
--(axis cs:0.599058582186227,0.396190350662095)
--(axis cs:0.594717577967486,0.398414313053781)
--(axis cs:0.590376573748745,0.400448697884038)
--(axis cs:0.586035569530004,0.40229164837748)
--(axis cs:0.581694565311264,0.403941577553799)
--(axis cs:0.577353561092523,0.405397168911865)
--(axis cs:0.573012556873782,0.406657376542332)
--(axis cs:0.568671552655041,0.40772142470452)
--(axis cs:0.564330548436301,0.408588806943639)
--(axis cs:0.55998954421756,0.409259284938693)
--(axis cs:0.555648539998819,0.409732887643684)
--(axis cs:0.551307535780078,0.410009912820621)
--(axis cs:0.546966531561337,0.410090941945422)
--(axis cs:0.542625527342597,0.4099769792185)
--(axis cs:0.538284523123856,0.409748035639934)
--(axis cs:0.533943518905115,0.409852807592519)
--(axis cs:0.529602514686374,0.409767026121407)
--(axis cs:0.525261510467634,0.409486086963702)
--(axis cs:0.520920506248893,0.409009307699526)
--(axis cs:0.516579502030152,0.408336357420114)
--(axis cs:0.512238497811411,0.407467182214818)
--(axis cs:0.507897493592671,0.406401995770221)
--(axis cs:0.50355648937393,0.40514127852397)
--(axis cs:0.499215485155189,0.403685778381791)
--(axis cs:0.494874480936448,0.402036511520787)
--(axis cs:0.490533476717707,0.40019476285767)
--(axis cs:0.486192472498967,0.398162086032949)
--(axis cs:0.481851468280226,0.395940302849463)
--(axis cs:0.477510464061485,0.393531502136119)
--(axis cs:0.473169459842744,0.390938038021945)
--(axis cs:0.468828455624004,0.388162527612809)
--(axis cs:0.464487451405263,0.385207848067474)
--(axis cs:0.460146447186522,0.382077133072766)
--(axis cs:0.455805442967781,0.378773768720157)
--(axis cs:0.45146443874904,0.375301388787922)
--(axis cs:0.4471234345303,0.371663869435353)
--(axis cs:0.442782430311559,0.367865323317185)
--(axis cs:0.438441426092818,0.363910093128115)
--(axis cs:0.434100421874077,0.359802744589302)
--(axis cs:0.429759417655337,0.355548058890185)
--(axis cs:0.425418413436596,0.351151024600778)
--(axis cs:0.421077409217855,0.346616829071141)
--(axis cs:0.416736404999114,0.341950849336406)
--(axis cs:0.412395400780374,0.337158642547109)
--(axis cs:0.408054396561633,0.332245935946264)
--(axis cs:0.403713392342892,0.327218616415848)
--(axis cs:0.399372388124151,0.32208271961691)
--(axis cs:0.39503138390541,0.316844418748872)
--(axis cs:0.39069037968667,0.311510012954936)
--(axis cs:0.386349375467929,0.306085915401761)
--(axis cs:0.382008371249188,0.30057864106345)
--(axis cs:0.377667367030447,0.294994794241631)
--(axis cs:0.373326362811707,0.289341055855931)
--(axis cs:0.368985358592966,0.283624170544149)
--(axis cs:0.364644354374225,0.277850933620037)
--(axis cs:0.360303350155484,0.272028177956943)
--(axis cs:0.355962345936743,0.266162760916841)
--(axis cs:0.351621341718003,0.260261551595168)
--(axis cs:0.347280337499262,0.254331419177079)
--(axis cs:0.342939333280521,0.24837922553545)
--(axis cs:0.33859832906178,0.242411840568087)
--(axis cs:0.33425732484304,0.236436439547645)
--(axis cs:0.329916320624299,0.230699538442958)
--(axis cs:0.325575316405558,0.225339969574329)
--(axis cs:0.321234312186817,0.219988654227756)
--(axis cs:0.316893307968076,0.214648400708018)
--(axis cs:0.312552303749336,0.209322684461575)
--(axis cs:0.308211299530595,0.204014937861208)
--(axis cs:0.303870295311854,0.19872850608088)
--(axis cs:0.299529291093113,0.193466638833734)
--(axis cs:0.295188286874373,0.188232488297775)
--(axis cs:0.290847282655632,0.183029108818061)
--(axis cs:0.286506278436891,0.177859457359989)
--(axis cs:0.28216527421815,0.172726394398149)
--(axis cs:0.27782426999941,0.167632685119465)
--(axis cs:0.273483265780669,0.162581000882657)
--(axis cs:0.269142261561928,0.157573920899351)
--(axis cs:0.264801257343187,0.152613934111949)
--(axis cs:0.260460253124446,0.147703441247353)
--(axis cs:0.256119248905706,0.142844757027576)
--(axis cs:0.251778244686965,0.138040112519388)
--(axis cs:0.247437240468224,0.133291657605334)
--(axis cs:0.243096236249483,0.128601463559)
--(axis cs:0.238755232030743,0.123971525707536)
--(axis cs:0.234414227812002,0.119403766164613)
--(axis cs:0.230073223593261,0.1149000366173)
--(axis cs:0.22573221937452,0.110462121150586)
--(axis cs:0.221391215155779,0.106091739093606)
--(axis cs:0.217050210937039,0.101790547871991)
--(axis cs:0.212709206718298,0.0975601458512955)
--(axis cs:0.208368202499557,0.0934020751569348)
--(axis cs:0.204027198280816,0.0893178244566398)
--(axis cs:0.199686194062076,0.0853088316925292)
--(axis cs:0.195345189843335,0.0813764867505686)
--(axis cs:0.191004185624594,0.0775221340570169)
--(axis cs:0.186663181405853,0.0737470750936124)
--(axis cs:0.182322177187112,0.0700525708275911)
--(axis cs:0.177981172968372,0.0664398440609923)
--(axis cs:0.173640168749631,0.0629100817228184)
--(axis cs:0.16929916453089,0.0594644371770191)
--(axis cs:0.164958160312149,0.0561040327657826)
--(axis cs:0.160617156093409,0.0528299633303127)
--(axis cs:0.156276151874668,0.0496433038617482)
--(axis cs:0.151935147655927,0.0465451414081808)
--(axis cs:0.147594143437186,0.0435369798657609)
--(axis cs:0.143253139218446,0.0409954815913579)
--(axis cs:0.138912134999705,0.0389082454502136)
--(axis cs:0.134571130780964,0.0369173237733807)
--(axis cs:0.130230126562223,0.0350189250666562)
--(axis cs:0.125889122343482,0.0332094891657693)
--(axis cs:0.121548118124742,0.0314853304304908)
--(axis cs:0.117207113906001,0.0298426190485532)
--(axis cs:0.11286610968726,0.0282773857743808)
--(axis cs:0.108525105468519,0.0267855310203227)
--(axis cs:0.104184101249779,0.025362835238315)
--(axis cs:0.0998430970310378,0.02400496983751)
--(axis cs:0.095502092812297,0.0227075083867643)
--(axis cs:0.0911610885935562,0.0214659379924676)
--(axis cs:0.0868200843748155,0.020275670789461)
--(axis cs:0.0824790801560747,0.0191320555007053)
--(axis cs:0.0781380759373339,0.0180303890291599)
--(axis cs:0.0737970717185931,0.0169659280489788)
--(axis cs:0.0694560674998524,0.0159339005652253)
--(axis cs:0.0651150632811116,0.0149295174124537)
--(axis cs:0.0607740590623708,0.0139479836636714)
--(axis cs:0.0564330548436301,0.0129845099220117)
--(axis cs:0.0520920506248893,0.0120343234683554)
--(axis cs:0.0477510464061485,0.0110926792392901)
--(axis cs:0.0434100421874077,0.0101548706111114)
--(axis cs:0.039069037968667,0.00921623996742042)
--(axis cs:0.0347280337499262,0.00827218903120852)
--(axis cs:0.0303870295311854,0.00731818894816373)
--(axis cs:0.0260460253124446,0.0063497901219823)
--(axis cs:0.0217050210937039,0.00536263183931091)
--(axis cs:0.0173640168749631,0.00435245183964077)
--(axis cs:0.0130230126562223,0.00331509645345428)
--(axis cs:0.00868200843748155,0.00224653449875049)
--(axis cs:0.00434100421874077,0.00114290377854194)
--(axis cs:0,1.96005085623076e-05)
--cycle;
\addlegendimage{area legend, fill=mediumseagreen85168104, fill opacity=0.5, line width=0.12pt}
\addlegendentry{95$\%$ confidence interval}

\addplot [line width=0.7pt, steelblue76114176]
table {%
0 5.09791486802413e-10
0.00434100421874077 0.000348197128365646
0.00868200843748155 0.000730309555572073
0.0130230126562223 0.00114832804027365
0.0173640168749631 0.00160427716545336
0.0217050210937039 0.00210021296191748
0.0260460253124446 0.00263821985246337
0.0303870295311854 0.00322040742883817
0.0347280337499262 0.00384890706429364
0.039069037968667 0.0045258683650653
0.0434100421874077 0.00525345546462036
0.0477510464061485 0.00603384316502181
0.0520920506248893 0.00686921293028464
0.0564330548436301 0.00776174873711333
0.0607740590623708 0.00871363278892046
0.0651150632811116 0.0097270410995452
0.0694560674998524 0.0108041389535848
0.0737970717185931 0.0119470762507701
0.0781380759373339 0.0131579827422788
0.0824790801560747 0.0144389631673973
0.0868200843748155 0.0157920922993712
0.0911610885935562 0.0172194099097727
0.095502092812297 0.0187229156611358
0.0998430970310378 0.0203045639380381
0.104184101249779 0.0219662586272169
0.108525105468519 0.0237098478576817
0.11286610968726 0.0255371187121678
0.117207113906001 0.027449791921595
0.121548118124742 0.029449516554529
0.125889122343482 0.0315378647139188
0.130230126562223 0.0337163262536609
0.134571130780964 0.0359863035277602
0.138912134999705 0.0383491061850718
0.143253139218446 0.0408059460227741
0.147594143437186 0.0433579319118677
0.151935147655927 0.0460060648080971
0.156276151874668 0.0487512328617682
0.160617156093409 0.0515942066399669
0.164958160312149 0.054535634474688
0.16929916453089 0.0575760379503455
0.173640168749631 0.0607158075440655
0.177981172968372 0.0639551984320377
0.182322177187112 0.0672943264750743
0.186663181405853 0.0707331643963129
0.191004185624594 0.0742715381637897
0.195345189843335 0.0779091235903355
0.199686194062076 0.0816454431629391
0.204027198280816 0.0854798631133934
0.208368202499557 0.0894115907416386
0.212709206718298 0.09343967200282
0.217050210937039 0.0975629893686004
0.221391215155779 0.101780259972795
0.22573221937452 0.106090034050851
0.230073223593261 0.11049069368215
0.234414227812002 0.114980451843499
0.238755232030743 0.119557351781563
0.243096236249483 0.124219266711318
0.247437240468224 0.128963899846941
0.251778244686965 0.133788784770792
0.256119248905706 0.138691286145465
0.260460253124446 0.143668600773055
0.264801257343187 0.148717759005037
0.269142261561928 0.153835626505346
0.273483265780669 0.15901890636838
0.27782426999941 0.164264141592848
0.28216527421815 0.169567717911511
0.286506278436891 0.174925866975965
0.290847282655632 0.180334669894792
0.295188286874373 0.185790061122447
0.299529291093113 0.191287832695429
0.303870295311854 0.196823638811329
0.308211299530595 0.20239300074548
0.312552303749336 0.207991312099065
0.316893307968076 0.213613844371588
0.321234312186817 0.219255752849816
0.325575316405558 0.224912082804367
0.329916320624299 0.230577775984304
0.33425732484304 0.236247677399239
0.33859832906178 0.241916542377635
0.342939333280521 0.247579043889213
0.347280337499262 0.253229780118559
0.351621341718003 0.258863282276319
0.355962345936743 0.264474022633621
0.360303350155484 0.270056422764688
0.364644354374225 0.275604861981943
0.368985358592966 0.281113685947285
0.373326362811707 0.286577215442642
0.377667367030447 0.291989755282352
0.382008371249188 0.297345603349417
0.386349375467929 0.302639059737219
0.39069037968667 0.307864435977888
0.39503138390541 0.313016064338112
0.399372388124151 0.318088307162866
0.403713392342892 0.323075566247291
0.408054396561633 0.327972292216697
0.412395400780374 0.332772993894496
0.416736404999114 0.33747224763778
0.421077409217855 0.34206470662015
0.425418413436596 0.346545110041431
0.429759417655337 0.350908292243896
0.434100421874077 0.355149191714775
0.438441426092818 0.359262859954908
0.442782430311559 0.363244470193653
0.4471234345303 0.367089325930389
0.45146443874904 0.370792869283253
0.455805442967781 0.37435068912613
0.460146447186522 0.377758528995306
0.464487451405263 0.381012294747672
0.468828455624004 0.384108061952852
0.473169459842744 0.387042083002218
0.477510464061485 0.389810793918329
0.481851468280226 0.392410820849012
0.486192472498967 0.394838986230957
0.490533476717707 0.397092314608494
0.494874480936448 0.399168038093921
0.499215485155189 0.401063601456637
0.50355648937393 0.402776666829131
0.507897493592671 0.404305118018789
0.512238497811411 0.405647064415385
0.516579502030152 0.40680084448507
0.520920506248893 0.407765028842618
0.525261510467634 0.408538422894719
0.529602514686374 0.409120069048083
0.533943518905115 0.409509248477155
0.538284523123856 0.409705482447314
0.542625527342597 0.409708533190424
0.546966531561337 0.409518404330732
0.551307535780078 0.409135340860138
0.555648539998819 0.408559828662929
0.55998954421756 0.407792593591181
0.564330548436301 0.406834600093036
0.568671552655041 0.405687049397203
0.573012556873782 0.404351377257985
0.577353561092523 0.402829251266283
0.581694565311264 0.401122567732946
0.586035569530004 0.399233448151921
0.590376573748745 0.397164235251602
0.594717577967486 0.394917488643742
0.599058582186227 0.392495980080233
0.603399586404967 0.389902688328968
0.607740590623708 0.387140793680854
0.612081594842449 0.384213672100909
0.61642259906119 0.381124889037162
0.620763603279931 0.377878192901841
0.625104607498671 0.374477508240074
0.629445611717412 0.370926928601993
0.633786615936153 0.367230709134782
0.638127620154894 0.363393258911787
0.642468624373635 0.359419133016381
0.646809628592375 0.355313024398733
0.651150632811116 0.351079755524127
0.655491637029857 0.346724269831817
0.659832641248598 0.342251623023802
0.664173645467338 0.33766697420317
0.668514649686079 0.332975576881891
0.67285565390482 0.32818276987817
0.677196658123561 0.323293968123533
0.681537662342302 0.318314653399975
0.685878666561042 0.313250365027479
0.690219670779783 0.308106690522185
0.694560674998524 0.302889256245447
0.698901679217265 0.297603718063836
0.703242683436005 0.292255752040009
0.707583687654746 0.286851045174089
0.711924691873487 0.281395286214955
0.716265696092228 0.275894156560474
0.720606700310968 0.270353321265372
0.724947704529709 0.264778420174956
0.72928870874845 0.259175059202501
0.733629712967191 0.253548801767547
0.737970717185932 0.247905160411835
0.742311721404672 0.242249588609006
0.746652725623413 0.236587472783563
0.750993729842154 0.230924124553947
0.755334734060895 0.225264773213867
0.759675738279635 0.219614558465305
0.764016742498376 0.213978523415894
0.768357746717117 0.208361607852542
0.772698750935858 0.202768641802425
0.777039755154598 0.197204339391615
0.781380759373339 0.191673293010813
0.78572176359208 0.186179967796752
0.790062767810821 0.180728696437033
0.794403772029562 0.175323674305225
0.798744776248302 0.169968954932226
0.803085780467043 0.164668445818955
0.807426784685784 0.1594259045946
0.811767788904525 0.154244935523704
0.816108793123265 0.149128986364558
0.820449797342006 0.144081345580398
0.824790801560747 0.139105139904137
0.829131805779488 0.134203332256394
0.833472809998228 0.129378720015827
0.837813814216969 0.124633933639862
0.84215481843571 0.119971435633167
0.846495822654451 0.11539351986034
0.850836826873192 0.110902311198586
0.855177831091932 0.106499765525333
0.859518835310673 0.102187670035045
0.863859839529414 0.0979676438787789
0.868200843748155 0.0938411391193459
0.872541847966895 0.0898094419943037
0.876882852185636 0.0858736744783818
0.881223856404377 0.0820347961363682
0.885564860623118 0.0782936062569201
0.889905864841859 0.0746507462572645
0.894246869060599 0.0711067023482599
0.89858787327934 0.0676618084488539
0.902928877498081 0.0643162493385617
0.907269881716822 0.0610700640362175
0.911610885935562 0.0579231493929279
0.915951890154303 0.0548752638868504
0.920292894373044 0.0519260316071717
0.924633898591785 0.049074946414446
0.928974902810526 0.0463213762642594
0.933315907029266 0.043664567681066
0.937656911248007 0.0411036503689206
0.941997915466748 0.0386376419457761
0.946338919685489 0.0362654527879824
0.950679923904229 0.0339858909716271
0.95502092812297 0.0317976672974085
0.959361932341711 0.0296994003858027
0.963702936560452 0.0276896218293981
0.968043940779193 0.0257667813894202
0.972384944997933 0.0239292522236377
0.976725949216674 0.0221753361330543
0.981066953435415 0.0205032688150237
0.985407957654156 0.0189112251106895
0.989748961872896 0.0173973242349402
0.994089966091637 0.0159596349773954
0.998430970310378 0.0145961808632647
1.00277197452912 0.013304945263299
1.00711297874786 0.0120838764424253
1.0114539829666 0.0109308925370675
1.01579498718534 0.00984388645157164
1.02013599140408 0.00882073066459393
1.02447699562282 0.00785928193676066
1.02881799984156 0.00695738591137024
1.0331590040603 0.00611288160039053
1.03750000827904 0.00532360574848201
1.04184101249779 0.00458739706827566
1.04618201671653 0.00390210034063293
1.05052302093527 0.00326557037411668
1.05486402515401 0.00267567581841555
1.05920502937275 0.00213030282696475
1.06354603359149 0.00162735856452456
1.06788703781023 0.00116477455597835
1.07222804202897 0.000740509873122297
1.07656904624771 0.000352554156713369
1.08091005046645 -1.06952845596586e-06
};
\addlegendentry{Mean prediction}
\end{axis}

\end{tikzpicture}

    }
    
    \caption{This is the caption of the graph}
    \label{fig:mygraph}
\end{figure}


  \end{block}

\end{column}

\separatorcolumn

\begin{column}{\colwidth}

  \begin{exampleblock}{A highlighted block containing some math}


  \end{exampleblock}

  \begin{block}{Calculated Rates for Every Possible Defect}

    Previous studies have calculated the ab initio energy barrier for every viable NnVH defect, however they have not used those results to fully calculate the reorientation rate of the Hydrogen when possible. The original ab inito results are compared to literature, along with a full calculation of the reorientation rate.

    \begin{table}
      \centering
      \begin{tabular}{l l c c c}
        \toprule
        \textbf{Defect}& &\textbf{Barrier / eV} & \textbf{Lit. / eV} & \textbf{Rate\,@\,0K / GHz} \\
        \midrule
        VH$^{0}$ &S$\,=\,3/2$ & 0.379 & 0.6 & 0\\
        NVH$^{0}$ &S$\,=\,0$ & 0.926 & 1.1 & 0\\
        NVH$^{0}$ &S$\,=\,1$ & 0.572 & 0.5 & 0\\
        NVH$^{-}$ &S$\,=\,1/2$ & 0.410 & 0.5 & 0\\
        NVH$^{+}$ &S$\,=\,1/2$ & 0.607 & 0.6 & 0\\
        N$_2$VH$^{0}$ &S$\,=\,1/2$ & 0.677 & 0.9 & 0\\
        N$_2$VH$^{-}$ &S$\,=\,0$ & 0.157 & 0.1 & 0\\
        N$_2$VH$^{+}$ &S$\,=\,0$ & 0.613 & 0.5 & 0\\

        \bottomrule
      \end{tabular}
      \caption{Comparison of the different quantum tunnelling energy barriers for different NnVH defects in diamond, along with their comparison to previous ab initio studies. The rates at two different temperatures are calculated for each defect with the methods previously shown.}
    \end{table}
  \end{block}

  \begin{block}{References}

    \nocite{*}
    \footnotesize{\bibliographystyle{plain}\bibliography{poster}}

  \end{block}

\end{column}

\separatorcolumn
\end{columns}
\end{frame}

\end{document}
